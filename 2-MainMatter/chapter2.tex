%!TEX root = ../template.tex
%%%%%%%%%%%%%%%%%%%%%%%%%%%%%%%%%%%%%%%%%%%%%%%%%%%%%%%%%%%%%%%%%%%%
%% chapter2.tex
%% NOVA thesis document file
%%
%% Chapter with the template manual
%%%%%%%%%%%%%%%%%%%%%%%%%%%%%%%%%%%%%%%%%%%%%%%%%%%%%%%%%%%%%%%%%%%%

\typeout{NT FILE chapter2.tex}%

\chapter{Literature review}
\label{cha:Literature review}

\glsresetall

\section{Sustainable Urban Logistics and the EV Transition}
\label{sec:Sustainable Urban Logistics and the EV Transition}

Driven by e-commerce and urbanization, last-minute delivery has grown substantially and it has intensified the challenges of urban freight distribution. As noted by Ahani et al. (2016), many studies have been published with the goal of dealing GHG, noise emission and congestion arising from the freight movements inside urban populated areas (Barter et al., 2012; Tipagornwong and Figliozzi, 2014; Dablanc, 2007; OECD, 2003; Pelletier et al., 2016; Russo and Comi, 2012). Urban logistics is a primary source of congestion and local pollution. Consequently, regulatory bodies are implementing stricter measures, such as Low Emission Zones (LEZ) and carbon taxes, forcing operators to take a second and think carefully about their fleet composition. \par

Electric Vehicles (EVs) have emerged as the leading alternative to Internal Combustion Engine Vehicles (ICEVs). Studies indicate that while EVs have higher acquisition costs, their operational costs (energy and maintenance) are significantly lower. However, the transition is not trivial. Authors like Nina (2010) and Pelletier et al. (2019) highlight that the "Total Cost of Ownership" (TCO) depends heavily on usage intensity, charging infrastructure availability, and battery lifespan uncertainty. Therefore, the decision to switch is not merely operational but a strategic investment under uncertainty. \par

Furthermore, the economic environment for fleet operators is becoming increasingly volatile. According to the World Bank’s Commodity Markets Outlook (2025), energy prices are subject to significant downside and upside risks driven by geopolitical tensions and supply chain disruptions. This report highlights that while supply conditions might stabilize some commodity prices, the risks of extreme fluctuations remain. For a fleet manager, this reinforces that relying on static cost averages is dangerous; a robust model must account for these potential extreme price swings (tail risks).

\section{Fleet Replacement Optimization Models}
\label{sec:Fleet Replacement Optimization Models}

The fleet replacement problem has been widely studied in Operations Research. Traditionally, these problems were modeled using deterministic approaches, assuming fixed future costs for fuel and vehicles. However, when considering real-world volatility, deterministic models are insufficient. \par

Recent literature has shifted towards stochastic optimization. For example, Onyshchenko et al. (2024) discuss integrated models for complex transport systems, emphasizing the need to account for variable parameters. Similarly, Ahani et al. (2023) introduced regulatory constraints into replacement models, showing that policy uncertainty significantly impacts optimal replacement timing. The consensus is clear: models must account for randomness in input parameters (fuel prices, demand, technology costs) to provide robust solutions. Kaszubowski (2019) proposes a three-tier qualitative method to evaluate urban freight models, emphasizing the need for tools that align strategic objectives with operational data. Similarly, a systematic review by Alvarez Gallo and Maheut (2023) highlights the extensive use of Multi-Criteria Decision Analysis (MCDA) to balance conflicting goals like sustainability and cost. However, while MCDA is excellent for ranking strategic alternatives, it often lacks the mathematical capacity to quantify financial risk distributions explicitly. This thesis moves beyond qualitative ranking to quantitative portfolio optimization, specifically addressing the financial risk of asset replacement described by Ahani et al. (2016).

\section{The Portfolio Approach in Transportation}
\label{sub:The Portfolio Approach in Transportation}

In order to handle these uncertainties, researchers have used and adapted the Modern Portfolio Theory (MPT), originally developed by Markowitz (1952) for financial markets. The core premise of the adapted MPT is that a fleet of vehicles can be treated as a portfolio of assets.

\begin{itemize}
    \item Assets: Vehicle technologies (Diesel, Electric, Hybrid).
    \item Return: The cost savings relative to a baseline.
    \item Risk: The volatility of these savings.
\end{itemize}

The work by Ahani et al. (2016) pioneered this approach for urban fleets. They formulated a Multi-Objective Optimization model where the objective was to minimize the Expected Total Cost (ETC) and minimize the Variance of the Total Cost. Their results demonstrated that a diversified fleet (mixing ICEVs and EVs) provides a hedge against fuel price spikes, similar to how a diversified stock portfolio protects against market crashes. This methodology provides the mathematical foundation for this thesis.

\section{Total Cost Formulation in Fleet Replacement Models}
\label{sub:Total Cost Formulation in Fleet Replacement Models}

In fleet replacement optimization problems, the definition of total cost plays a central role in evaluating alternative fleet compositions and investment strategies. Following the framework proposed by Ahani et al. (2016), the total cost of a fleet over the planning horizon is typically composed of several cost components, including vehicle acquisition costs, energy costs, maintenance and operation expenses, and residual values at the end of the vehicle life cycle.

Vehicle acquisition costs represent the upfront capital expenditure associated with purchasing new vehicles of different technologies. Energy costs include fuel or electricity consumption and depend on vehicle efficiency, usage intensity, and energy prices, which may vary over time and across scenarios. Maintenance and operating costs capture routine servicing, repairs, and other operational expenses that are influenced by vehicle technology and age. In some formulations, residual values or replacement costs are also considered to account for the remaining value of vehicles at the end of the planning horizon.

When uncertainty is present, some components of the total cost—most notably energy prices and operating expenses—are treated as stochastic variables. In this case, the total cost becomes a random variable, whose expected value and variance can be computed across a finite set of scenarios. This representation enables the integration of total cost modeling with portfolio-based optimization approaches, where Expected Total Cost is minimized while controlling for cost variability.

Let $X_{i,t,k}$ denote the number of vehicles of technology $k$ and age $i$ operated in period $t$, $Z_{t,k}$ the number of new vehicles of technology $k$ purchased at the beginning of period $t$, and $Y_{i,t,k}$ the number of vehicles of technology $k$ and age $i$ salvaged at the end of period $t$. The total cost formulation consists of the following components.

%% ------------------------- TOTAL ENERGY COST -------------------------
Energy Cost: The total energy cost incurred by operating the fleet is given by:
\begin{equation}
FC = \sum_{i=0}^{A_k-1}\sum_{t=0}^{T-1}\sum_{k=1}^{K}
f_{i,t,k}\,u_{i,k}\,X_{i,t,k}\,(1+dr)^{-t},
\label{eq:FC}
\end{equation}
where $f_{i,t,k}$ represents the unit energy cost per kilometer for a vehicle of age $i$ and technology $k$ in period $t$, $u_{i,k}$ is the annual utilization, and $dr$ denotes the discount rate.

%% ----------------------- OPEX -------------------------
Maintenance and operating costs are modeled as:
\begin{equation}
MC = \sum_{i=0}^{A_k-1}\sum_{t=0}^{T-1}\sum_{k=1}^{K}
m_{i,t,k}\,u_{i,k}\,X_{i,t,k}\,(1+dr)^{-t},
\label{eq:MC}
\end{equation}
where $m_{i,t,k}$ denotes the maintenance cost per kilometer of vehicle k of an age i during period t per (€/km)..

%% ----------------------- CAPEX -------------------------
The capital investment cost associated with acquiring new vehicles is expressed as:
\begin{equation}
CIC = \sum_{t=0}^{T-1}\sum_{k=1}^{K}
v_{k,t}\,Z_{t,k}\,(1+dr)^{-t},
\label{eq:CIC}
\end{equation}
where $v_{k,t}$ is the purchase cost (€) per unit of a vehicle of technology $k$ in period $t$.

%% ----------------------- EMISSION COST -------------------------
In addition, emission-related costs can be incorporated as:
\begin{equation}
EC = \sum_{i=0}^{A_k-1}\sum_{t=0}^{T-1}\sum_{k=1}^{K}
e_{i,k}\,u_{i,k}\,X_{i,t,k}\,(1+dr)^{-t},
\label{eq:EC}
\end{equation}
where $e_{i,k}$ represents CO$_2$ emission cost (€/km) of vehicle of age $i$, type $k$.

%% ----------------------- SALVAGE REVENUE -------------------------
Finally, the salvage revenue obtained from retiring vehicles at the end of their service life is given by:
\begin{equation}
SR = \sum_{i=1}^{A_k}\sum_{t=0}^{T}\sum_{k=1}^{K}
s_{i,k}\,Y_{i,t,k}\,(1+dr)^{-t},
\label{eq:SR}
\end{equation}
where $s_{i,k}$ denotes the salvage revenue (€) of a vehicle of age $i$ and technology $k$.

%% ----------------------- TOTAL COST -------------------------
By combining all cost components, the total cost of the fleet over the planning horizon can be expressed as:
\begin{equation}
TC = \sum_{t=0}^{T-1}\sum_{k=1}^{K} v_{k,t}\,Z_{t,k}\,(1+dr)^{-t}
- \sum_{i=1}^{A_k}\sum_{t=0}^{T}\sum_{k=1}^{K} s_{i,k}\,Y_{i,t,k}\,(1+dr)^{-t}
\nonumber\\
\quad + \sum_{i=0}^{A_k-1}\sum_{t=0}^{T-1}\sum_{k=1}^{K}
\bigl(f_{i,t,k}+m_{i,t,k}+e_{i,k}\bigr)\,u_{i,k}\,X_{i,t,k}\,(1+dr)^{-t}.
\label{eq:TC}
\end{equation}

When uncertainty is present in parameters such as energy prices, maintenance costs, or vehicle prices, the total cost $TC$ becomes a random variable. This representation allows the Expected Total Cost and the variance of total cost to be computed across a finite set of scenarios, providing the basis for mean-variance optimization in fleet replacement decisions.

By adopting the total cost formulation proposed by Ahani et al. (2016), this thesis ensures consistency with established models in the literature while providing a solid foundation for extending the analysis to more detailed fleet representations and multi-objective optimization frameworks.


\section{Risk Modeling in Mean–Variance Fleet Optimization}
\label{sub:Risk Modeling in Mean–Variance Fleet Optimization}

Based on the total cost formulation described in the previous section, uncertainty in urban fleet replacement decisions can be explicitly represented through the variability of total cost outcomes. When key cost components such as energy prices, maintenance expenses, and vehicle acquisition costs are uncertain, the total cost of operating the fleet becomes a stochastic quantity. In this context, risk is commonly quantified using the variance of total cost, which measures the dispersion of cost realizations around their expected value and captures the degree of cost volatility faced by decision-makers.

\subsection{Cost Uncertainty in Urban Fleet Replacement}
\label{sub:Cost Uncertainty in Urban Fleet Replacement}

Urban freight fleet replacement decisions are characterized by significant uncertainty arising from multiple cost components. Key sources of uncertainty include fuel and electricity prices, vehicle acquisition costs, maintenance and repair expenses, and residual values at the end of a vehicle’s service life. These cost elements are influenced by external factors such as energy market volatility, technological progress, regulatory changes, and macroeconomic conditions, making long-term cost estimation inherently uncertain.

In this context, relying solely on deterministic or average cost values can lead to misleading conclusions and suboptimal investment decisions. Instead, fleet replacement models must explicitly account for uncertainty in order to provide solutions that are robust to fluctuations in key parameters. This has motivated the use of stochastic optimization approaches in the literature, where uncertainty is represented through scenarios or probability distributions and incorporated directly into the decision-making process.

\subsection{Variance as a Measure of Risk}
\label{sub:Variance as a Measure of Risk}

Within stochastic optimization and portfolio-based approaches, risk is commonly quantified through the variance of total cost. Variance measures the dispersion of cost outcomes around their expected value and captures the degree of cost volatility faced by decision-makers. In the context of urban freight fleets, a higher variance indicates greater exposure to unpredictable cost fluctuations, which may challenge budget planning and financial stability.

The use of variance as a risk measure is well established in both financial economics and transportation research. Its mathematical properties make it particularly attractive for optimization models, as it is intuitive, widely understood, and compatible with quadratic or linearized formulations. Moreover, variance provides a transparent way to represent risk preferences by allowing decision-makers to balance expected cost minimization against acceptable levels of cost variability.

In fleet replacement problems, variance reflects the combined uncertainty associated with different vehicle technologies and energy sources. By diversifying the fleet composition across technologies with distinct cost structures and uncertainty profiles, operators can reduce overall cost volatility, in line with the principles of Modern Portfolio Theory.

\subsection{Mean–Variance Formulation in Fleet Replacement Models}
\label{sub:Mean–Variance Formulation in Fleet Replacement Models}
Building on the principles of Modern Portfolio Theory, the mean–variance approach has been adapted to urban fleet replacement problems by modeling vehicle technologies as portfolio assets with uncertain costs. In this framework, the Expected Total Cost (ETC) represents the mean performance criterion, while the variance of total cost captures the associated risk. The fleet replacement problem is thus formulated as a multi-objective optimization problem that seeks to minimize both ETC and cost variance.

A seminal application of this approach in urban freight transportation is provided by Ahani et al. (2016), who demonstrated that a diversified fleet composed of conventional and alternative fuel vehicles can hedge against uncertainty in fuel prices and operating costs. Their results showed that the mean–variance framework allows decision-makers to explore the trade-offs between cost efficiency and risk exposure, leading to a set of Pareto-optimal fleet configurations rather than a single optimal solution.

From a modeling perspective, the mean–variance formulation is particularly suitable for mixed-integer linear programming (MILP) extensions of fleet replacement problems. When cost uncertainty is represented through a finite set of scenarios, the expected cost and variance can be computed explicitly and incorporated into the optimization model. This enables the systematic exploration of risk–cost trade-offs using established multi-objective solution techniques, such as the e(needstochange)-constraint method, while maintaining computational tractability.

Overall, the mean–variance framework provides a coherent and well-founded approach for modeling risk in urban fleet replacement decisions. It allows the explicit representation of uncertainty, supports diversification across vehicle technologies, and offers decision-makers a transparent tool to balance economic performance and risk in the transition toward more sustainable urban freight systems.


% % subsection with_a_local_latex_installation (end)

% \subsection{With a Remote Cloud-based Service} % (fold)
% \label{sub:with_a_remote_cloud_based_service}

% Follow these steps to get started with a remote \LaTeX\ installation:

% \begin{itemize}
%   \item Download the \href{https://github.com/joaomlourenco/novathesis/archive/main.zip}{latest version from the GitHub repository as a Zip file}.
%   \item Login to your favorite LaTeX cloud service. I recommend \href{https://www.overleaf.com/?r=f5160636&rm=d&rs=b}{Overleaf} but there are alternatives. These instructions apply to Overleaf and you'll have to adapt for other providers.
%   \item In the menu select \fbox{New project}$\rightarrow$\fbox{Upload project}.
%   \item Select “\verb!template.tex!” as the main file.
%   \item Follow from Step~\ref{it:project_available} above in Section~\ref{sub:with_a_local_latex_installation} (\nameref{sub:with_a_local_latex_installation}).
% \end{itemize}

% % subsection with_a_remote_cloud_based_service (end)


% \section{Folder and Files}
% \label{sec:folders_and_files}

% The \gls{novathesis} template is organized into many files and folders. At the main level it includes the following files and folders listed in Table~\ref{tab:folders_and_files}.

% \newcommand{\accessAllowed}{\includegraphics[align=c,width=1.9em]{access_allowed}}
% \newcommand{\accessForbiden}{\includegraphics[align=c,width=1.9em]{dont_touch}}
% \newcommand{\File}{\includegraphics[align=c,width=1.9em]{file}}
% \newcommand{\Folder}{\includegraphics[align=c,width=1.9em]{folder}}


% \bgroup
%     \rowcolors{1}{}{GhostWhite}
%       \begin{xltabular}{\textwidth}{>{\ttfamily}l>{\itshape}lcX}
%         \caption{The folders and files (top level).}
%         \label{tab:folders_and_files}\\
%         \toprule
%         \rowcolor{Gainsboro}%
%         Name & Type & Access & Contents \\
%         \midrule
% template.tex      & \File    & \accessForbiden &
% The main template file. You need to \emph{compile} this file with one of \pdfLaTeX, \XeLaTeX, or \LuaLaTeX\ to obtain the PDF file (”\texttt{template.pdf}”).  I recommend the usage of the ”\texttt{latexmk}” command or, if you use a UN*X-like OS, you may use ”\texttt{make}” (and the given ”\texttt{Makefile}”).
% \\
% Config          & \Folder  & \accessAllowed &
% Configuration files.  Please customize your template by changing the files in this folder!
% \\
% Chapters          & \Folder  & \accessAllowed &
% Examples of document contents, including Chapters, Appendices, Annexes, Abstracts, Glossaries, Lists of Symbols, etc. Replace them with your own.
% \\
% Bibliography      & \Folder    & \accessAllowed &
% Where all your bibliography files should be located. You may have has many bibliography files as you want.
% \\
% template.pdf      & \File    & \accessAllowed &
% A possible result of applying \pdfLaTeX\ to the “\texttt{template.tex}” file. The look and feel of the document will depend on the parametriza\-tion/\-con\-fig\-u\-ra\-tion (e.g., School) of this template.
% \\
% novathesis.cls     & \File    & \accessForbiden &
% The main class file.
% \\
% NOVAthesisFiles   & \Folder  & \accessForbiden &
% Additional files for the \gls{novathesis} template.  This is where all the juice is so, unless you are a \TeX magician, don't mess up with the files and folders inside this folder.
% \\
%         \bottomrule
%         \end{xltabular}
%     % \end{longtblr}
% \egroup

% \bgroup
%     \rowcolors{1}{}{GhostWhite}
%       \begin{xltabular}{\textwidth}{>{\ttfamily}l>{\itshape}lcX}
%         \caption{The configuration files (\texttt{Config} folder).}
%         \label{tab:configuration_files}\\
%         \toprule
%         \rowcolor{Gainsboro}%
%         Name & Type & Access & Contents \\
%         \midrule
% \texttt{0\_memoir.tex}      & \File  & \accessForbiden &
% Options specific for the \texttt{memoir} class. \emph{Don't touch this file unless you know what you are doing!}
% \\
% \texttt{1\_novathesis.tex}  & \File  & \accessAllowed &
% The main configuration file for the template, e.g., select the document type, the school, the used languages, etc.  
% \\
% \texttt{2\_biblatex.tex}      & \File  & \accessAllowed &
% Select how your citations and bibliographic references will be printed.  The default is numbers inside square brackets, e.g. \cite{novathesis-manual}, but you can change it to other formats, such as author-year, e.g., \citeauthor{novathesis-manual}~(\citeyear{novathesis-manual}).
% \\
% \texttt{3\_cover.tex}		& \File & \accessAllowed & 
% Configure cover contents (e.g., author's name, thesis/dissertation title, author, advisers, committee, etc)
% \\
% \texttt{4\_files.tex}		& \File & \accessAllowed & 
% Select which files shall be included in the document as chapters, appendices, annexes, etc…
% \\
% \texttt{5\_packages.tex}		& \File & \accessAllowed & 
% User's customization, such as loading additional packages and declare user defined commands.
% \\
% \texttt{6\_list\_of.tex}		& \File & \accessForbiden & 
% Configure the lists to be printed (table of contents, list of figures, list of tables, list of listings, etc). \emph{Don't touch this file unless you know what you are doing!}
% \\
% \texttt{9\_nova\_fct.tex}	& \File & \accessAllowed & 
% Configurations specific to NOVA FCT. \emph{Otherwise ignored.}
% \\
% \texttt{9\_ulisboa\_fmv.tex}	& \File & \accessAllowed & 
% Configurations specific to ULISBOA FMV. \emph{Otherwise ignored.}
% \\
% \texttt{9\_ulisboa\_ist.tex}	& \File & \accessAllowed & 
% Configurations specific to ULISBOA IST. \emph{Otherwise ignored.}
% \\
% \texttt{9\_uminho.tex}		& \File & \accessAllowed & 
% Configurations specific to UMINHO (all schools). \emph{Otherwise ignored.}
% \\
%         \bottomrule
%         \end{xltabular}
%     % \end{longtblr}
% \egroup


% % section folder_structure (end)

% % ===================
% % = Package options =
% % ===================
% \section{Customizing the \novathesistxt\ template}
% \label{sec:package_options}

% The \novathesistxt\ template can be customized by editing the files in the \texttt{Config} folder.

% \newcommand{\classoption}[4]{\textbf{#1=OPT}\newline\emph{\small#2}&\textbf{#3}\newline{\small#4}\\}
% \newcommand{\defaultopt}[1]{\mbox{$\Rightarrow$~\emph{Default: \texttt{#1}}}\newline}
% \newcommand{\defaultit}[1][default]{($\Leftarrow$~\emph{#1})}


% \subsection{Options in \texttt{1\_novathesis.tex}} % (fold)
% \label{sub:_texttt_1__novathesis_tex}

% \subsubsection{Most Relevant Options (\texttt{1\_novathesis.tex})} % (fold)
% \label{ssub:most_relevant_options}

% \bgroup
% \begin{xltabular}{\linewidth}{>{\hsize=.4\hsize\raggedright\arraybackslash}X>{\hsize=.6\hsize}X}
%   \toprule
% %----------------------------------------------------------------------
%   \classoption{doctype}%
%     {phd, phdprop, phdplan, msc, mscplan, bsc, plain}%
%     {The type of the document.}%
% 	{%
%     \begin{tabular}{@{}r@{ $\rightarrow$ }l@{}}
%         phd & PhD thesis \defaultit.\\
%     phdprop & PhD thesis proposal (for NOVA FCT).\\
%     phdplan & PhD thesis plan.\\
%         msc & MSc thesis.\\
%     mscplan & MSc thesis plan.\\
%         bsc & BSc report.\\
%       plain & Other report.\\
%     \end{tabular}
%     }
% %----------------------------------------------------------------------
%     \midrule
%   \classoption{school}%
%   	{nova/fct, nova/fcsh, nova/ims, nova/ims/mcsig, nova/ims/mgt, nova/ensp, nova/itqb/green, nova/itqb/gray,
% ulisboa/ist, ulisboa/fcul, ulisboa/fmv,
% uminho/eaad, uminho/ec, uminho/ed, uminho/eeg, uminho/eeng, uminho/elach, uminho/emed, uminho/epsi, uminho/ese, uminho/i3bs, uminho/ics, uminho/ie, 
% iscteiul/eta, 
% ips/ests, 
% ipl/isel, ipl/isel/meb,
% ulht/deisi, ulht/mge, 
% other/esep
% 	}%
%     {Selection of the university and of the school (and degree variant).}%
%     {\defaultopt{school=nova/fct} 
%      This option changes the typesetting of the de document to some specific School formating and layout, like covers, margins, fonts, paragraph spacing and indentation, etc.}
% %----------------------------------------------------------------------
%     \midrule
%   \classoption{docstatus}%
%     {draft, provisional, final}%
%     {The current status of the document.}%
% 	{%
%     \begin{tabular}{@{}r@{ $\rightarrow$ }X@{}}
%          working      & Working version \defaultit.\\
%          provisional  & Version for submission.\\
%          final        & Final version.\\
%     \end{tabular}
%     }
% %----------------------------------------------------------------------
%     \midrule
%   \classoption{lang}%
%     {en, pt, de, es, fr, gr, it}%
%     {The main language for the document.}%
% 	{%
%     \begin{tabular}{@{}l@{ $\rightarrow$ }X@{}}
%          en & Enlgish \defaultit.\\
%          pt & Portuguese.\\
%          de & German.\\
%          es & Spanish.\\
%          fr & French.\\
%          gr & Greek.\\
%          it & Italian.\\
%     \end{tabular}
%     }
% %----------------------------------------------------------------------
%     \midrule
%   \classoption{media}%
%     {screen, paper}%
%     {The target media for the PDF.}%
% 	{%
%     \begin{tabular}{@{}l@{ $\rightarrow$ }X@{}}
%          screen & No empty/white pages \defaultit.\\
%          paper  & Empty/white pages are added when necessary.\\
%     \end{tabular}
%     }
% %----------------------------------------------------------------------
%     \midrule
%   \classoption{print/webography}%
%     {User defined title}%
%     {Generate a separate bibliography for \emph{@online} references.}%
% 	{%
% 		\defaultopt{print/webography=undefined} 
% 		If undefined, the \emph{@online} references are list in the main bibliography.  If defined, the \emph{@online} references will be printed in a separate bibliography titled as given in the option.
%     }
% %----------------------------------------------------------------------
%     \midrule
%   \classoption{color/links}%
%     {Color name}%
%     {The color for the hyperlinks (URLs, cross references, citations).}%
% 	{%
% 		\defaultopt{color/links=DarkBlue} 
% 		The valid color names as listed in “\texttt{xcolor}” manual, the “\texttt{svgname}” color set.
%     }
% %----------------------------------------------------------------------
%     \midrule
%   \classoption{glossaries/color/gls}%
%     {Color name}%
%     {The color for the glossary managed hyperlinks (glossary, symbols, etc).}%
% 	{%
% 		\defaultopt{glossaries/color/gls=Black} 
% 		The valid color names as listed in “\texttt{xcolor}” manual, the “\texttt{svgname}” color set.
%     }
% %----------------------------------------------------------------------
%     \midrule
%   \classoption{print/index}%
%     {true,\newline false \defaultit}%
%     {Print the (words) index at the end of the document.}%
% 	{%
% 		Print the index (in Portuguese \emph{Índice Remissivo}).
%     }
% %----------------------------------------------------------------------
%     \bottomrule
% \end{xltabular}
% \egroup


% \subsubsection{Less Relevant Options (\texttt{1\_novathesis.tex})} % (fold)
% \label{ssub:less_relevant_options_1}


% \bgroup
% \begin{xltabular}{\linewidth}{>{\hsize=.4\hsize\raggedright\arraybackslash}X>{\hsize=.6\hsize}X}
%   \toprule
% %----------------------------------------------------------------------
% 	  \classoption{abstractorder}%
% 	    {$L_0 = \{L_1, L_2, \ldots, L_n\}$}%
% 	    {Forces the abstracts languages and order for documents in language $L_0$.}%
% 		{%
% 		\defaultopt{abstractorder=\{en=\{en,pt\}\} for english}
% 		\defaultopt{abstractorder=\{L=\{L,en\}\} for lang L}
% 		$L_i$ is a two-letters language code from the set of valid language codes, following ISO 3166-1 (alfa-2).
% 	    }
% %----------------------------------------------------------------------
% 	    \midrule
% 	  \classoption{lang/extra}%
% 	    {$\{L_1, L_2, \ldots, L_n\}$}%
% 	    {List of additional languages are used in the document besides the main laguage and those used in the abstracts (above).}%
% 		{%
% 		\defaultopt{lang/extra=\{\}}
% 		$L_i$ is a two-letters language code from the set of valid language codes, following ISO 3166-1 (alfa-2).
% 	    }
% %----------------------------------------------------------------------
% 	    \midrule
% 	  \classoption{glossaries/list/reverse}%
%   	  	{true, \defaultit\newline false}%
% 	    {Shall the glossary entries list the page numbers where those entries are used?  (Like a reverse index!)}%
% 		{}
% %----------------------------------------------------------------------
% 	    \midrule
% 	  \classoption{tocintoc}%
%   	  	{true,\newline false \defaultit}%
% 	    {Shall table of contents be listed in the table of contents?}%
% 		{}
% %----------------------------------------------------------------------
% 	    \midrule
% 	  \classoption{tocintoc}%
%   	  	{true,\newline false \defaultit}%
% 	    {Shall a second cover page be forced?}%
% 		{If the contents for the second page are not defined, the second cover will be a replica of the first cover.}
% %----------------------------------------------------------------------
% 	    \midrule
% 	  \classoption{print/committee}%
%   	  	{true,\newline false}%
% 	    {Shall the evaluation committee be printed?}%
% 		{%
% 		\defaultopt{print/committee=true if docstatus=final}
% 		\defaultopt{print/committee=false otherwise}
% 	    }
% %----------------------------------------------------------------------
% 	    \midrule
% 	  \classoption{print/statement}%
%   	  	{true,\newline false \defaultit}%
% 	    {Shall the honor/originality statement be printed?}%
% 		{}
% %----------------------------------------------------------------------
% 	    \midrule
% 	  \classoption{print/copyright}%
%   	  	{true, \defaultit\newline false}%
% 	    {Shall the copyright message be printed?}%
% 		{}
% %----------------------------------------------------------------------
% 	    \midrule
% 	  \classoption{print/timestamp}%
%   	  	{true,\newline false}%
% 	    {Shall a timestamp (with the PDF generation date/time) be printed in the cover?}%
% 		{%
% 		\defaultopt{print/timestamp=true if docstatus=working}
% 		\defaultopt{print/timestamp=false otherwise}
% 		}
% %----------------------------------------------------------------------
% 	    \midrule
% 	  \classoption{style/url}%
%   	  	{default, \defaultit\newline same}%
% 	    {Use the same (main) font in URLs?}%
% 		{}
% %----------------------------------------------------------------------
%     \midrule
%   \classoption{style/font}%
%     {arial, bookman, calibri, erewhon, kieranhealy, kpfonts, libertine, newpx, newsgott, scholax}%
%     {Which font set to use in the document?}%
% 	{%
%     \begin{tabular}{@{}l@{ $\rightarrow$ }X@{}}
% 		arial 		& Use `arial' font. Requires Xe\LaTeX\ or Lua\LaTeX.\\
% 		bookman 	& Use `bookman' font.\\
% 		calibri 	& Use `calibri' font. Requires Xe\LaTeX\ or Lua\LaTeX.\\
% 		erewhon 	& Use `erewhon' font.\\
% 		kieranhealy & Use `kieranhealy' font.\\
% 		kpfonts 	& Use `palatino' font.\\
% 		libertine 	& Use `libertine' font.\\
% 		newpx 		& Use `palatino' font. \defaultit\\
% 		newsgott 	& Use `newsgott' font. Requires Xe\LaTeX\ or Lua\LaTeX.\\
% 		scholax 	& Use `scholax' font.\\
%     \end{tabular}
%     }
% %----------------------------------------------------------------------
% 	    \midrule
% 	  \classoption{style/chapter}%
% 	    {\emph{See list on the side!}}%
% 	    {Which chapter style to use in the document?}%
% 	{%
% 	Besides the standard \href{https://tug.ctan.org/info/MemoirChapStyles/MemoirChapStyles.pdf}{\texttt{memoir} chapter styles} (\texttt{default}, \texttt{section}, \texttt{article}, \texttt{reparticle}, \texttt{hangnum}, \texttt{companion}, \texttt{demo}, \texttt{bianchi}, \texttt{bringhurst}, \texttt{brotherton}, \texttt{chappell}, \texttt{culver}, \texttt{dash}, \texttt{demodemoell}, \texttt{ger}, \texttt{lyhne}, \texttt{madsen}, \texttt{pedersen}, \texttt{southall}, \texttt{thatcher}, \texttt{veelo}, \texttt{verville}, \texttt{crosshead}, \texttt{dowding}, \texttt{komalike}, \texttt{ntglike}, \texttt{tandh}, \texttt{wilsondob}), the customized list of chapter styles below is also available.
% 	    \begin{tabular}{@{}l@{ $\rightarrow$ }X@{}}
% 		bar 		& Use `bar' chapter style. \defaultit\\
% 		bar-compact	& Use `bar-compact' chapter style.\\
% 		bluebox 	& Use `bluebox' chapter style.\\
% 		compact 	& Use `compact' chapter style.\\
% 		elegant 	& Use `elegant' chapter style.\\
% 		fmv 		& Use `fmv' chapter style.\\
% 		hansen 		& Use `hansen' chapter style.\\
% 		ist 		& Use `ist' chapter style.\\
% 		ist2 		& Use `ist2' chapter style.\\
% 		pedersen 	& Use `pedersen' chapter style.\\
% 	    \end{tabular}
% 	    }
% %----------------------------------------------------------------------
%     \midrule
%   \classoption{lang/cover}%
%     {en, pt, de, es, fr, gr, it}%
%     {The main language for the cover.}%
% 	{%
% 	\defaultopt{The same as the main language.}
%     }
% %----------------------------------------------------------------------
%     \midrule
%   \classoption{lang/copyright}%
%     {en, pt, de, es, fr, gr, it}%
%     {The main language for the copyright message.}%
% 	{%
% 	\defaultopt{The same as the main language.}
%     }
% %----------------------------------------------------------------------
%     \midrule
%   \classoption{spine/layout}%
%     {no, full, trim}%
%     {Print the “book spine” at the end of the document?}%
% 	{%
% 		\defaultopt{`trim' if docstatus=final}
% 		\defaultopt{`no' otherwise}
% 	    \begin{tabular}{@{}l@{ $\rightarrow$ }X@{}}
% 		no 		& do not print the book spine.\\
% 		full	& print the book spine in a full page.\\
% 		trim 	& print and trim the page to the width of the book spine.\\
% 	    \end{tabular}
%     }
% %----------------------------------------------------------------------
%     \midrule
%   \classoption{spine/width}%
%     {\emph{\LaTeX\ dimension}}%
%     {Force the width of the “book spine”.}%
% 	{%
% 		\defaultopt{The “natural width”.}
% 		 The defailt width for the book spine will be the width of the number of pages of the document if printed in standard paper ($80g/m^2$).
%     }
% %----------------------------------------------------------------------
%     \midrule
%   \classoption{debug}%
%     {cover, spine}%
%     {Activate debug mode for cover and/or book spine.}%
% 	{%
% 		\defaultopt{debug=\{\}}
%     }
% % %----------------------------------------------------------------------
% 	  %   \midrule
% 	  % \classoption{linkscolor}%
% 	  %   {A color of your choice.}%
% 	  %   {The color for all the hyperlinks in the PDF file.}%
% 	  %   {\defaultopt{darkblue}
% 	  %    The “\texttt{media=paper}” option (see below) will override this option to “\texttt{black}”}
% % %----------------------------------------------------------------------
% %     \midrule
% %   \classoption{media}%
% %     {screen, paper}%
% %     {The target of the PDF.}%
% %     {\defaultopt{screen}
% %      By default, PDF for screen has colored links and identical left and right margins, while PDF for paper (to print) has black links and may have different left and right margins.}
% %     \midrule
% % %----------------------------------------------------------------------
% %   \classoption{print/index}%
% %     {true, false}%
% %     {Produce the document index.}%
% %     {\defaultopt{false}
% %      The index (\emph{índice remissivo}) is a keyword index typeset an the end of the document. WARNING: Should not be confused with the table of contents.}
% %     \midrule
% %
% %
% %
% %
% % %----------------------------------------------------------------------
% %   \classoption{fontstyle}%
% %     {bookman, charter, fourier, kpfonts(*), mathpazo1, mathpazo2, newcent}%
% %     {The font set to be used in the document.}{Please note that a font set include definitions for the main text, headings, maths, etc.}
% %     \midrule
% % %----------------------------------------------------------------------
% %   \classoption{chapstyle}%
% %     {bianchi, bluebox, brotherton, dash, default, elegant(*), ell, ger, hansen, ist, jenor, lyhne, madsen, pedersen, veelo, vz14, vz34, vz43}%
% %     {The chapter style}{The look of the chapter beginning.}
% %     \midrule
% % %----------------------------------------------------------------------
% %   \classoption{converlang}%
% %     {en, pt(*)}%
% %     {The language to be used when typesetting the cover page.}{}
% %     \midrule
% % %----------------------------------------------------------------------
% %   \classoption{otherlistsat}%
% %     {front(*), back}%
% %     {Where to put the other lists besides the table of contents.}{The default is (\texttt{front}) before the main text.  But some scientific areas prefer them at the end of the document (\texttt{back}), just before the Appendixes.}
% %     \midrule
% % %----------------------------------------------------------------------
% %   \classoption{statement}%
% %     {true, false(*)}%
% %     {Include or don't include the contents of the “\texttt{statement}” file.}{The default is for this file to be ignored (if it exists).}
% %     \midrule
% % %----------------------------------------------------------------------
% %   \classoption{spine}%
% %     {true, false(*)}%
% %     {Generate the book spine and the last page in the PDF.}{}
% %     \midrule
% % %----------------------------------------------------------------------
% %   \classoption{biblatex}%
% %     {OPT=\{list of options for \texttt{biblatex}\}}%
% %     {Customize \texttt{biblatex}, the bibliography management system used in this class.}{Probably you will want to change the value of the \texttt{biblatex} “\texttt{style}” option. For other customizations of \texttt{biblatex} check its manual.}
% %     \midrule
% % %----------------------------------------------------------------------
% %   \classoption{memoir}%
% %     {OPT=\{list of options for \texttt{memoir}\}}%
% %     {Customize the base class \texttt{memoir}.}{The \texttt{memoir} manual should be the first document to be consulted when looking for “\textbf{how can I do this?}” You may what to change the base font size from 11pt to a smaller (10pt) or larger (12pt) size.  Also, remember to change the “\texttt{draft}” to final when your document is finished.}
% %     \midrule
%     \bottomrule
% \end{xltabular}
% \egroup
% % \end{ctabular}


% \section{How to Write Using \LaTeX}
% \label{sec:how_to_write_using_latex}

% Please have a look at Chapter~\ref{cha:a_short_latex_tutorial_with_examples}, where you may find many examples of \LaTeX constructs, such as Sectioning, inserting Figures and Tables, writing Equations, Theorems and algorithms, exhibit code listings, etc.

% % section how_to_write_using_latex (end)



% \section{Example glossary, acronyms, and symbols}
% %
% % \todo[inline]{A a note in a line by itself.}
% %
% This is the first occurrence of an abbreviation: \gls{abbrev}. And now the second occurrence of the same abbreviation: \gls{abbrev}. And a new acronym with capital letter: \Gls{xpt} and reused \gls{xpt}.  Let's also use a few other acronyms such as \gls{aaa}, \gls{aab}, \gls{aba}, \gls{bbb} and \gls{xpt}.
% In geometry, the area enclosed by a circle of radius \gls{r} is $\pi r^2$. Here the Greek letter \gls{pi} is equal to the ratio of the circumference of any circle to its diameter.
% Lets add ``\gls{computer}'' to the glossary! Be carefull with mathematical symbols in acronyms, please see the definition of \gls{mu}.

% % Reference to Potassium \gls{chem:potassio} and Sodium \gls{chem:sodio} as well.

% %
% % Please note that
% % \begin{center}
% %   \textbf{\large this package and template are not official for FCT/NOVA}.
% % \end{center}



% % \printbibliography[heading=subbibliography, segment=\therefsegment, title={\bibname\ for chapter~\thechapter}]


% \endinput

% %!TEX root = ../template.tex
% %%%%%%%%%%%%%%%%%%%%%%%%%%%%%%%%%%%%%%%%%%%%%%%%%%%%%%%%%%%%%%%%%%%%
% %% chapter2.tex
% %% NOVA thesis document file
% %%
% %% Chapter with the template manual
% %%%%%%%%%%%%%%%%%%%%%%%%%%%%%%%%%%%%%%%%%%%%%%%%%%%%%%%%%%%%%%%%%%%%

% \typeout{NT FILE chapter2.tex}%

% \chapter{NOVAthesis Template \emph{User's Manual}}
% \label{cha:users_manual}

% \glsresetall

% \begin{center}
%   \fbox{\LARGE
%     This manual is outdated and must be revised!}
% \end{center}

% Referência ao Potássio é \gls{chem:potassio} e Sódio também \gls{chem:sodio}.

% \section{Introduction}
% \label{sec:introduction}

% This chapter describes how to use the \gls{novathesis}\ Template and the \gls{novathesisclass} file.  I will assume you have a working installation of \LaTeX, wither local (in your own computer) or remote (in Overleaf), and that it compiled successfully the default configuration (PhD for \gls{FCT}).


% \section{Folder Structure}
% \label{sec:folder_structure}

% The \gls{novathesis} template is organized into many files and folders. At the main level it includes the following files and folders:

% \noindent
% \bgroup
% \rowcolors{1}{GhostWhite}{}
% \begin{xltabular}{\linewidth}{>{\ttfamily}l>{\itshape}l>{\upshape}X}
% novathesis.cls     & file    &
% The main class file. It will include additional files from \texttt{NOVAthesisFiles} folder and its sub-folders.
% \\
% template.tex      & file    &
% The main template file. You need to \emph{compile} this file with one of pdf\LaTeX, \XeLaTeX, or \LuaLaTeX\ to obtain the \texttt{template.pdf} file.
% \\
% bibliography.bib  & file    &
% An example of a bibliography file. You may have has many as you want. \\
% template.pdf      & file    &
% A possible result of applying pdf\LaTeX\ to the \texttt{template.tex} file. The template supports multiple types of documents (e.g., MSc dissertation, PhD thesis, …) and multiple Schools (e.g., NOVA FCT, FCSH-NOVA, IST-UL, FC-UL, …) and each will produce different results.
% \\
% Chapters          & folder  & Examples of thesis chapters. Replace them with your own chapters.
% \\
% Examples          & folder  & Some more examples of the use of the template for different document types and Schools.
% \\
% Scripts           & folder  & Some (possibly useful) scripts for Unix-based systems (Linux, Mac OSx). If you are a windows user, ignore this folder (you may safely delete it if you want).
% \\
% NOVAthesisFiles   & folder  &
% Additional files for the \gls{novathesisclass}\ file.  Unless you know what you are doing, avoid messing up with the files and folders inside this folder (except for deleting the unused Schools, see below).
% \\
% \end{xltabular}
% \egroup

% The \texttt{NOVAthesisFiles} folder contains additional files and folders that complement the main \gls{novathesisclass}\ file.  These are:

% \noindent
% \bgroup
% \rowcolors{1}{GhostWhite}{}
% \begin{tabularx}{\linewidth}{>{\ttfamily}l>{\itshape}l>{\upshape}X}
% README.txt      & file    &
% A file that should be read!  :)
% \\
% fix-babel.tex   & file    &
% Simple fixes to the \texttt{babel} package.
% \\
% lang-text.ldf   & file    &
% Translations of important strings used in the template.  Currently fully supported are Portuguese and English, but French is on the way.  If you add translations for your own language, please be so kind and send them to me. Thank you!
% \\
% options.tex     & file    &
% Processing of \gls{novathesisclass}\ options.  \emph{Don't mess with this!}
% \\
% packages.tex    & file    &
% Additional packages to be loaded into the \gls{novathesis}\ template. \emph{You should not mess with this!}
% \\
% spine.tex       & file    &
% This file is loaded only if the option \texttt{spine=full} or \texttt{spine=trim}, and includes the typesetting of the book spine.
% \\
% ChapStyles      & folder  &
% Contains a lot of files, one for each chapter style.  If you really know what you are doing, you may add your own chapter style here.
% \\
% FontStyles      & folder  &
% Contains a few files, one for each set of fonts (main text font, chapter font, section font, subsection font, etc).  If you really know what you are doing, you may add your own set here.
% \\
% Schools         & folder  &
% Configuration files for each school.  This folder is organized into subfolders, one for each university.  \emph{You may safely delete all the subfolders except the one for your University.}  Then open the subfolder of your University and \emph{you may safely delete all the subfolders except the one for your School/Faculty}.
% \\
% \end{tabularx}
% \egroup

% As stated above, the \texttt{Schools} folder contains per-university folders and per-school (faculty) subfolders.  Currently these are the available folders:

% \noindent
% \bgroup
% \rowcolors{1}{GhostWhite}{}
% \begin{tabularx}{\linewidth}{>{\ttfamily}r@{~/~}>{\ttfamily}l>{\itshape}l>{\upshape}X}
% ul     & ist    & folder  &
% The folder for the \href{http://www.tecnico.ulisboa.pt}{\emph{Instituto Superior Técnico}} of the \emph{University of Lisbon}.
% \\
% nova    & fcsh   & folder  &
% The folder for the \href{http:www.fcsh.unl.pt}{\emph{Faculty of Human and Social Sciences}}  of the \emph{NOVA University of Lisbon}.
% \\
% nova    & fct    & folder  &
% The folder for the \href{http:www.fct.unl.pt}{\emph{Faculty of Sciences and Technology}} of the \emph{NOVA University of Lisbon}.
% \\
% nova    & novaims    & folder  &
% The folder for the \href{http:www.novaims.unl.pt}{\emph{Information and Management School}} of the \emph{NOVA University of Lisbon}.
% \\
% \end{tabularx}
% \egroup

% % section folder_structure (end)

% % ===================
% % = Package options =
% % ===================
% \section{\glsfmtshort{novathesisclass}\ Class Options}
% \label{sec:package_options}

% The \gls{novathesisclass}\ can be customized with the options listed below.

% \newcommand{\classoption}[3]{\textbf{#1=OPT}\qquad #2\\\qquad\emph{#3}\\}

% \noindent
% \begin{ctabular}{@{}p{\linewidth}@{}}
%   \toprule
% %----------------------------------------------------------------------
%   \classoption{doctype}%
%     {phd(*), phdplan, phdprop, msc, mscplan, bsc}%
%     {The type of the document: PhD Thesis (---~Default~---), PhD Plan, PhD Proposal, MSc Dissertation, MSc Plan, BSc Report}
%     \midrule
% %----------------------------------------------------------------------
%   \classoption{school}%
%     {nova/fct(*), nova/fcsh, nova/ims, ul/ist, ul/fc}%
%     {The name of the school. This option changes the typesetting of the cover and some School specific formating, like margins, fonts, paragraph spacing and indentation, etc…}
%     \midrule
% %----------------------------------------------------------------------
%   \classoption{lang}%
%     {en(*), pt}%
%     {The main language for the document.  Currently only Portuguese and English are supported.  Other languages are expected to be support in forthcoming versions.}
%     \midrule
% %----------------------------------------------------------------------
%   \classoption{fontstyle}%
%     {bookman, charter, fourier, kpfonts(*), mathpazo1, mathpazo2, newcent}%
%     {The font set to be used in the document.  Please note that a font set include definitions for the main text, headings, maths, etc.}
%     \midrule
% %----------------------------------------------------------------------
% %----------------------------------------------------------------------
%   \classoption{chapstyle}%
%     {bianchi, bluebox, brotherton, dash, default, elegant(*), ell, ger, hansen, ist, jenor, lyhne, madsen, pedersen, veelo, vz14, vz34, vz43}%
%     {The chapter style, i.e., the look of the chapter beginning.}
%     \midrule
% %----------------------------------------------------------------------
%   \classoption{converlang}%
%     {en, pt(*)}%
%     {The language to be used when typesetting the cover page.}
%     \midrule
% %----------------------------------------------------------------------
%   \classoption{otherlistsat}%
%     {front(*), back}%
%     {Where to put the other lists besides the table of contents. The default is (\texttt{front}) before the main text.  But some scientific areas prefer them at the end of the document (\texttt{back}), just before the Appendixes.}
%     \midrule
% %----------------------------------------------------------------------
%   \classoption{statement}%
%     {true, false(*)}%
%     {Include or don't include the contents of the “\texttt{statement}” file. The default is for this file to be ignored (if it exists).}
%     \midrule
% %----------------------------------------------------------------------
%   \classoption{linkscolor}%
%     {darkblue(*), black}%
%     {The color for all the hyperlinks in the PDF file.  The “\texttt{media=paper}” option (see below) will override this option to “\texttt{black}”}
%     \midrule
% %----------------------------------------------------------------------
%   \classoption{spine}%
%     {true, false(*)}%
%     {Generate the book spine and the last page in the PDF.}
%     \midrule
% %----------------------------------------------------------------------
%   \classoption{biblatex}%
%     {OPT=\{list of options for \texttt{biblatex}\}}%
%     {Customize \texttt{biblatex}, the bibliography management system used in this class. Probably you will want to change the value of the \texttt{biblatex} “\texttt{style}” option. For other customizations of \texttt{biblatex} check its manual.}
%     \midrule
% %----------------------------------------------------------------------
%   \classoption{memoir}%
%     {OPT=\{list of options for \texttt{memoir}\}}%
%     {Customize the base class \texttt{memoir}. The \texttt{memoir} manual should be the first document to be consulted when looking for “\textbf{how can I do this?}” You may what to change the base font size from 11pt to a smaller (10pt) or larger (12pt) size.  Also, remember to change the “\texttt{draft}” to final when your document is finished.}
%     \midrule
% %----------------------------------------------------------------------
%   \classoption{media}%
%     {screen(*), paper}%
%     {Behavior to be customized in the school options/configuration. Expected definitions for screen are: left and right margins are equal and use colored links. Expected definitions for paper are: left and right margins are different and use black links.}
%     \bottomrule
% \end{ctabular}

% \section{Additional considerations about the class options}
% \label{sec:additional_considerations}

% In this section we will provide some additional considerations about some of the customizations available as class options.

% \subsection{The main language}
% \label{sub:the_main_language}

% The choice of the main language with the option “\texttt{lang=OPT}” affects:

% \begin{itemize}
%   \item \textbf{The order of the summaries.} First is printed the abstract in the main language and then in the foreign language. This means that if your main language for the document in English, you will see first the “abstract” (in English) and then the “resumo” (in Portuguese). If you switch the main language for the document for Portuguese, it will also automatically switch the order of the summaries to “resumo” and then “abstract”.
%   \item \textbf{The names for document sectioning.} E.g., ``Chapter'' vs.\ ``Capítulo'', ``Table of Contents'' vs.\ ``Índice'', ``Figure'' vs.\ ``Figura'', etc.
%   \item \textbf{The type of documents in the bibliography.} E.g., ``Technical Report'' vs.\ ``Relatório Técnico'', ``PhD Thesis'' vs.\ ``Tese de Doutoramento'', etc.
% \end{itemize}

% No mater which language you chose, you will always have the appropriate hyphenation rules according to the language at that point. You always get Portuguese hyphenation rules in the ``Resumo'', English hyphenation rules in the ``Abstract'', and then the main language hyphenation rules for the rest of the document.

% % subsection the_main_language (end).

% % section additional_consideration (end)


% \subsection{Class of Text}
% \label{sub:class_of_text}

% You must choose the class of text for the document. The available options are:

% \begin{enumerate}
%   \item \textbf{bsc} --- BSc graduation report.
%   \item \textbf{*mscplan} --- Preparation of MSc dissertation. This is a preliminary report graduate students at DI-NOVA FCT must prepare to conclude the first semester of the two-semesters MSc work. The files specified by \verb!\ntdedicationfile! and \verb!\acknowledgmentsfile! are ignored, even if present, for this class of document.
%   \item \textbf{msc} --- MSc dissertation.
%   \item \textbf{phdprop} ---  Proposal for a PhD work. The files specified by \verb!\ntdedicationfile! and \verb!\acknowledgmentsfile! are ignored, even if present, for this class of document.
%   \item \textbf{prepphd} ---  Preparation of a PhD thesis. This is a preliminary report PhD students at DI-NOVA FCT must prepare before the end of the third semester of PhD work. The files specified by \verb!\ntdedicationfile! and \verb!\acknowledgmentsfile! are ignored, even if present, for this class of document.
%   \item \textbf{phd} --- PhD dissertation.
% \end{enumerate}
% % subsection class_of_text (end)

% % ============
% % = Printing =
% % ============
% \subsection{Printing}
% \label{sub:printing}

% You must choose how your document will be printed. The available options are:
% \begin{enumerate}
%   \item \textbf{oneside} --- Single side page printing.
%   \item \textbf{*twoside} --- Double sided page printing.
% \end{enumerate}
% % subsection printing (end)

% % =============
% % = Font Size =
% % =============
% \subsection{Font Size}
% \label{ssec:font_size}

% You must select the encoding for your text. The available options are:
% \begin{enumerate}
%   \item \textbf{11pt} --- Eleven (11) points font size.
%   \item \textbf{*12pt} --- Twelve (12) points font size. You should really stick to 12pt\ldots
% \end{enumerate}
% % subsection font_size (end)

% % =================
% % = Text encoding =
% % =================
% \subsection{Text Encoding}
% \label{ssec:text_encoding}

% You must choose the font size for your document. The available options are:
% \begin{enumerate}
%   \item \textbf{latin1} --- Use Latin-1 (\href{http://en.wikipedia.org/wiki/ISO/IEC_8859-1}{ISO 8859-1}) encoding.  Most probably you should use this option if you use Windows;
%   \item \textbf{utf8} --- Use \href{http://en.wikipedia.org/wiki/UTF-8}{UTF8} encoding.    Most probably you should use this option if you are not using Windows.
% \end{enumerate}
% % subsection font_size (end)

% % ============
% % = Examples =
% % ============
% \subsection{Examples}
% \label{ssec:examples}

% Let's have a look at a couple of examples:

% \begin{itemize}
%   \item Preparation of PhD thesis, in portuguese, with 11pt size and to be printed single sided (I wonder why one would do this!)\\
%   \verb!\documentclass[prepphd,pt,11pt,oneside,latin1]{thesisdifct-nova}!
%   \item MSc dissertation, in English, with 12pt size and to be printed double sided\\
%   \verb!\documentclass[msc,en,12pt,twoside,utf8]{thesisdifct-nova}!
% \end{itemize}
% % subsection examples (end)

% \section{How to Write Using \LaTeX}
% \label{sec:how_to_write_using_latex}

% Please have a look at Chapter~\ref{cha:a_short_latex_tutorial_with_examples}, where you may find many examples of \LaTeX constructs, such as Sectioning, inserting Figures and Tables, writing Equations, Theorems and algorithms, exhibit code listings, etc.

% % section how_to_write_using_latex (end)



% \section{Example glossary, acronyms, and symbols}
% %
% % \todo[inline]{A a note in a line by itself.}
% %
% This is the first occurrence of an abbreviation: \gls{abbrev}. And now the second occurrence of the same abbreviation: \gls{abbrev}. And a new acronym with capital letter: \Gls{xpt} and reused \gls{xpt}.  Let's also use a few other acronyms such as \gls{aaa}, \gls{aab}, \gls{aba}, \gls{bbb} and \gls{xpt}.
% In geometry, the area enclosed by a circle of radius \gls{r} is $\pi r^2$. Here the Greek letter \gls{pi} is equal to the ratio of the circumference of any circle to its diameter.
% Lets add ``\gls{computer}'' to the glossary! Be carefull with mathematical symbols in acronyms, please see the definition of \gls{mu}.
% %
% % Please note that
% % \begin{center}
% %   \textbf{\large this package and template are not official for FCT/NOVA}.
% % \end{center}



% % \printbibliography[heading=subbibliography, segment=\therefsegment, title={\bibname\ for chapter~\thechapter}]
