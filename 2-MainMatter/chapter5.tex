% %!TEX root = ../template.tex
% %%%%%%%%%%%%%%%%%%%%%%%%%%%%%%%%%%%%%%%%%%%%%%%%%%%%%%%%%%%%%%%%%%%%
% %% chapter5.tex
% %% NOVA thesis document file
% %%
% %% Chapter with lots of dummy text
% %%%%%%%%%%%%%%%%%%%%%%%%%%%%%%%%%%%%%%%%%%%%%%%%%%%%%%%%%%%%%%%%%%%%

% \typeout{NT FILE chapter5.tex}%

% \chapter{Future Work}
% \label{cha:Future Work}

% \section{Tasks}
% \label{sec:Tasks}
% Future work in this dissertation will include the following tasks:
% \newpage
% \textbf{Task 1: Data Collection and Scenario Generation.} Collection of historical and projected data regarding vehicle acquisition costs (ICEVs, HEVs and EVs), energy prices (fuel and electricity), and maintenance costs. Based on this data, I will use Python to code the total cost formulation and the mean–variance objectives defined in Chapter 4, along with all operational and fleet balance constraints. The ε-constraint method will be applied to generate a set of Pareto-optimal solutions representing different trade-offs between Expected Total Cost and cost Variance. \par

% Computational experiments will be conducted to assess model feasibility, convergence behavior, and solution quality under different parameter settings. These experiments will serve as the basis for evaluating the practical applicability of the proposed framework. \par

% \textbf{Task 2: Analysis of Cost–Risk Trade-offs.} Once the efficient frontier is obtained, the next step involves analyzing the economic trade-offs between cost efficiency and risk exposure. The impact of increasing risk aversion on fleet composition will be examined by comparing solutions associated with different variance thresholds. This analysis will provide insights into how diversification across vehicle technologies contributes to reducing cost volatility while maintaining acceptable expected costs. \par

% \textbf{Task 3: Sensitivity and Elasticity Analysis.} Conducting the robustness analysis defined in Research Question 4. I will systematically vary critical parameters (e.g., electricity prices, battery degradation costs, and carbon taxes) to calculate the elasticity of the fleet composition. The goal is to identify the economic "tipping points" where the optimal decision shifts significantly from ICEVs to EVs.\par

% \textbf{Task 4: Policy-Oriented Scenario Evaluation.} The proposed framework will also be used to explore policy-driven scenarios relevant to urban freight operations. These may include changes in energy taxation, technology-specific incentives, or access restrictions associated with low-emission zones. By adjusting cost parameters and constraints accordingly, the model can be used to assess how different policy environments influence fleet replacement decisions and the associated cost–risk trade-offs.

% \textbf{Task 5: Model Extensions.} As a final step, potential extensions of the model will be explored, subject to time and data availability. These extensions may include a more detailed representation of fleet heterogeneity, such as multiple vehicle sizes or differentiated usage profiles, as well as alternative planning horizons. While not essential for the core objectives of this dissertation, such extensions would further enhance the realism and applicability of the proposed framework.

% \textbf{Task 6: Writing the Dissertation.} Compilation of all chapters, including the literature review, methodology, discussion of results, and conclusions. This task also includes the formatting of the document according to the university's standards and the preparation for the final defense.

% \lipsum[1-100]
% \lipsum[1-100]
% \lipsum[1-100]
% % \lipsum[1-100]
% % \lipsum[1-700]
% % \lipsum[1-700]
% % \lipsum[1-700]
% % \lipsum[1-700]
% % \lipsum[1-700]
% % \lipsum[1-700]
% % \lipsum[1-700]
% % \lipsum[1-700]
